\chapter{Bazy danych}
\section{Normalizacja baz danych – pierwsza, druga i trzecia postać normalna.}
Нормализация базы данных - это процесс проектирования структуры базы данных с целью уменьшить дублирование данных 
и избегать проблем с добавлением, удалением и модификацией данных. 
В этом процессе используются правила или "формы", известные как нормальные формы.

Первая нормальная форма (1NF):
\begin{itemize}
\item Все колонки должны быть атомарными, то есть каждое значение в колонке должно быть неделимым.
\item Каждая колонка должна иметь уникальное имя.
\item Все строки должны быть уникальными, не должно быть дубликатов.
\end{itemize}

Вторая нормальная форма (2NF):
\begin{itemize}
\item Таблица уже находится в 1NF.
\item Все колонки, не являющиеся ключами, полностью зависят от первичного ключа. Иначе говоря, не должно быть такой ситуации, когда значение колонки, не являющейся ключом, зависит от одной части составного ключа.
\end{itemize}

Третья нормальная форма (3NF):
\begin{itemize}
\item Таблица уже находится в 2NF.
\item Не существует транзитивных зависимостей. То есть ни одна неключевая колонка не должна зависеть от других неключевых колонок. Все зависимости должны исходить только от ключа.
\end{itemize}

Имеются и более высокие уровни нормализации, такие как BCNF, 4NF, 5NF и 6NF, каждый из которых решает определенные проблемы в проектировании баз данных. Однако первые три нормальные формы наиболее важны и часто используются для большинства приложений баз данных

\section{Modele baz danych (logiczny, relacyjny, fizyczny).}

\subsection*{relacyjny}
W terminologii matematycznej - baza danych jest zbiorem relacji.
Stąd historycznie pochodzi nazwa relacyjny model danych i relacyjna baza danych. W
matematyce definiuje się relację jako podzbiór iloczynu kartezjańskiego zbiorów wartości.
Reprezentacją relacji jest dwuwymiarowa tabela złożona z kolumn i wierszy.

Założenia modelu relacyjnego:

\begin{itemize}
\item Liczba kolumn/atrybutów/pól (synonimy) jest z góry ustalona.
\item Z każdą kolumną jest związana jej nazwa (np. FirstName) oraz dziedzina (np.TEXT(20)), określająca zbiór wartości, jakie mogą wystąpić w kolumnie.
\item Na przecięciu wiersza/krotki/rekordu (synonimy) i kolumny znajduje się pojedyncza (atomowa) wartość należąca do dziedziny kolumny
\item  Wiersz reprezentuje jeden rekord informacji np. osobę.
\item W modelu relacyjnym abstrahujemy od kolejności wierszy (rekordów) i kolumn (pól w rekordzie).
\end{itemize}

Klucz główny: dla każdej tabeli musi być określony klucz główny, będący jednoznacznym identyfikatorem. Może to być jedna lub więcej kolumn, w których wartości jednoznacznie identyfikują cały wiersz. Klucz główny w tabeli może być tylko jeden.
Klucz jednoznaczny ma te same właściwości co klucz główny, ale ich może być w tabeli
więcej niż jeden.
Klucz obcy - jedna lub więcej kolumn, których wartości występują również jako klucz
główny/jednoznaczny w tej samej/innej tabeli i są interpretowane jako wskaźniki do wierszy w tej drugiej tabeli


Логическая модель базы данных описывает данные так, как они видны пользователям. 
Это абстракция, которая помогает упростить взаимодействие пользователя с базой данных. 
В логической модели определяются сущности (таблицы), их атрибуты (поля) и связи между сущностями.

Например, в базе данных компании мы можем определить сущности "Сотрудники", 
"Отделы" и "Проекты". Сущность "Сотрудники" может иметь атрибуты "Имя", "Фамилия", 
"Дата рождения", "Позиция", и т.д. Сущность "Отделы" может включать "Название отдела", 
"Местоположение", и т.д. Затем, связи между этими сущностями определяют, 
как они взаимодействуют друг с другом, например, один отдел может включать множество сотрудников.
реляцийная это прокаченная логическая модель

Физическая модель базы данных, с другой стороны, описывает, как данные фактически хранятся в системе, включая информацию о физическом хранении и доступе к данным.
Физическая модель базы данных может включать следующие детали:
Способ хранения данных: это может быть последовательное хранение, индексное хранение, хеширование или другие методы.
Детали файловой системы: в каких файлах и на каких дисках хранятся данные, как они организованы и так далее.
Методы индексации: используются ли индексы для ускорения поиска данных, какие индексы используются, как они структурированы и так далее.
Компрессия данных: используется ли сжатие данных для экономии места, какой метод сжатия используется и так далее.
Шифрование данных: если данные шифруются для защиты, как это делается, какой алгоритм шифрования используется и так далее.

Физическая модель базы данных — это подробное представление о том, как данные хранятся в компьютерной памяти или на диске. 
Она описывает структуру хранения, методы доступа, пути обращения к данным и так далее. Физическая модель обычно не доступна для пользователей базы данных, 
она служит для оптимизации работы СУБД (системы управления базами данных).

\section{Rodzaje zapytań w języku SQL.}

SQL (Structured Query Language) является стандартным языком для работы с реляционными базами данных. Он позволяет выполнять различные виды запросов для манипулирования и извлечения данных. В SQL есть несколько основных типов запросов:

Запросы выборки данных (SELECT):
\begin{verbatim}
SELECT column1, column2, ...
FROM table_name;
\end{verbatim}
Эти запросы используются для извлечения данных из базы. Вы можете выбрать одну или несколько колонок, и применять различные условия и операции сортировки.

Запросы вставки данных (INSERT INTO):
\begin{verbatim}
INSERT INTO table_name (column1, column2, column3, ...)
VALUES (value1, value2, value3, ...);
\end{verbatim}
С помощью этих запросов можно добавить новые строки в таблицу.

Запросы обновления данных (UPDATE):
\begin{verbatim}
UPDATE table_name
SET column1 = value1, column2 = value2, ...
WHERE condition;
\end{verbatim}
UPDATE позволяет изменять значения в уже существующих строках таблицы.

Запросы удаления данных (DELETE):
\begin{verbatim}
DELETE FROM table_name WHERE condition;
\end{verbatim}
DELETE используется для удаления строк из таблицы.

Запросы создания новой таблицы (CREATE TABLE):
\begin{verbatim}
CREATE TABLE table_name (
column1 datatype,
column2 datatype,
column3 datatype,
....
);
\end{verbatim}
CREATE TABLE создает новую таблицу с заданными именами столбцов и типами данных.

Запросы изменения структуры таблицы (ALTER TABLE):
\begin{verbatim}
ALTER TABLE table_name
ADD column_name datatype;
\end{verbatim}
ALTER TABLE позволяет добавлять, удалять или изменять столбцы в существующей таблице.


\section{Funkcje w języku SQL.}

В SQL имеются встроенные функции, которые позволяют выполнять расчеты, трансформации и манипуляции с данными. Они могут быть разделены на следующие категории:

Агрегатные функции (Aggregate Functions) - функции, работающие с множеством строк и возвращающие одно значение:
\begin{itemize}
\item AVG() - вычисляет среднее значение набора чисел.
\item COUNT() - подсчитывает количество строк в выборке.
\item MAX() - находит максимальное значение в наборе данных.
\item MIN() - находит минимальное значение в наборе данных.
\item SUM() - суммирует значения в наборе чисел.
\end{itemize}

Функции для работы со строками (String Functions) - функции, предназначенные для манипулирования строками:
\begin{itemize}
\item CONCAT() - соединяет две или более строк в одну.
\item LENGTH() - возвращает длину строки.
\item LOWER() и UPPER() - переводит строку в нижний или верхний регистр соответственно.
\item TRIM() - удаляет пробелы из начала и конца строки.
\end{itemize}

% Функции для работы с датами (Date Functions) - функции, которые позволяют манипулировать и извлекать информацию из дат:
% \begin{itemize}
%     \item CURRENT_DATE() - возвращает текущую дату.
%     \item DATE_ADD() и DATE_SUB() - добавляет или вычитает определенное количество дней к дате.
%     \item DAY(), MONTH(), YEAR() - извлекает день, месяц или год из даты соответственно.
% \end{itemize}

Функции NULL-значений (NULL Functions) - функции, предназначенные для работы с NULL-значениями:
\begin{itemize}
\item ISNULL() - проверяет, является ли значение NULL.
\item COALESCE() - возвращает первое не-NULL значение из списка.
\end{itemize}

\section{Transakcje w bazach danych. }


Транзакция в контексте баз данных — это логическая единица работы, которая может включать одну или несколько операций над данными (например, чтение, вставка, обновление, удаление).

Транзакции важны для обеспечения надежности и целостности данных. Они обеспечивают выполнение следующих четырех свойств, известных как свойства ACID:

Атомарность (Atomicity): Это свойство гарантирует, что транзакция считается одной неделимой единицей работы. Это означает, что либо все операции в транзакции выполняются успешно, либо, если хотя бы одна операция не удается, не выполняется ни одна операция.
Согласованность (Consistency): Согласованность обеспечивает, что транзакция приводит базу данных из одного согласованного состояния в другое. После транзакции все ограничения целостности в базе данных должны быть сохранены.
Изолированность (Isolation): Это свойство гарантирует, что параллельные транзакции не влияют друг на друга. Результат параллельного выполнения транзакций должен быть таким же, как если бы эти транзакции выполнялись последовательно.
Долговечность (Durability): После того, как транзакция успешно завершена, ее результаты становятся постоянными и остаются в базе данных даже в случае сбоя системы.
В SQL транзакции управляются с помощью команд BEGIN TRANSACTION, COMMIT и ROLLBACK. Например:


BEGIN TRANSACTION;
INSERT INTO table1 (column1) VALUES ('value1');
UPDATE table2 SET column2 = 'value2' WHERE column3 = 'value3';
COMMIT;

Если в процессе транзакции что-то идет не так, можно использовать ROLLBACK, чтобы отменить все изменения, внесенные в рамках этой транзакции.