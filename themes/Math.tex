\chapter{Math}
\section{Wektory i macierze – definicje i podstawowe operacje. }

\subsection{Векторы}

Вектор -- это элемент векторного пространства, что является обобщением понятия ``направленного отрезка'', известного из геометрии. Векторы обычно используются для представления величин, имеющих и размер, и направление.

\[
\vec{v} = (v_1, v_2, ..., v_n)
\]

\subsection{Матрицы}

Матрица -- это прямоугольная таблица чисел, символов или выражений, упорядоченная по строкам и столбцам. Числа, символы или выражения называются элементами матрицы.

\[
A = \begin{bmatrix}
a_{11} & a_{12} & \dots & a_{1n} \\
a_{21} & a_{22} & \dots & a_{2n} \\
\vdots & \vdots & \ddots & \vdots \\
a_{m1} & a_{m2} & \dots & a_{mn}
\end{bmatrix}
\]

\subsection{Основные операции}

С векторами и матрицами можно выполнять различные операции. 

\subsubsection{Сложение и вычитание векторов}

\[
\vec{v} + \vec{u} = (v_1 + u_1, v_2 + u_2, ..., v_n + u_n)
\]

\[
\vec{v} - \vec{u} = (v_1 - u_1, v_2 - u_2, ..., v_n - u_n)
\]

\subsubsection{Умножение вектора на скаляр}

\[
\alpha \vec{v} = (\alpha v_1, \alpha v_2, ..., \alpha v_n)
\]

\subsubsection{Скалярное произведение векторов}

\[
\vec{v} \cdot \vec{u} = v_1u_1 + v_2u_2 + \dots + v_nu_n
\]

\subsubsection{Сложение и вычитание матриц}

\[
A + B = \begin{bmatrix}
a_{11} + b_{11} & a_{12} + b_{12} & \dots & a_{1n} + b_{1n} \\
a_{21} + b_{21} & a_{22} + b_{22} & \dots & a_{2n} + b_{2n} \\
\vdots & \vdots & \ddots & \vdots \\
a_{m1} + b_{m1} & a_{m2} + b_{m2} & \dots & a_{mn} + b_{mn}
\end{bmatrix}
\]

\[
A - B = \begin{bmatrix}
a_{11} - b_{11} & a_{12} - b_{12} & \dots & a_{1n} - b_{1n} \\
a_{21} - b_{21} & a_{22} - b_{22} & \dots & a_{2n} - b_{2n} \\
\vdots & \vdots & \ddots & \vdots \\
a_{m1} - b_{m1} & a_{m2} - b_{m2} & \dots & a_{mn} - b_{mn}
\end{bmatrix}
\]

\subsubsection{Умножение матрицы на скаляр}

\[
\alpha A = \begin{bmatrix}
\alpha a_{11} & \alpha a_{12} & \dots & \alpha a_{1n} \\
\alpha a_{21} & \alpha a_{22} & \dots & \alpha a_{2n} \\
\vdots & \vdots & \ddots & \vdots \\
\alpha a_{m1} & \alpha a_{m2} & \dots & \alpha a_{mn}
\end{bmatrix}
\]

\subsubsection{Умножение матриц}

Пусть $A$ -- матрица размера $m \times n$ и $B$ -- матрица размера $n \times p$, тогда произведение $AB$ будет матрицей размера $m \times p$.

\[
(AB)_{ij} = \sum_{k=1}^n a_{ik}b_{kj}
\]



\section{Definicja funkcji obliczalnej (częściowo rekurencyjnej).}

Функция называется \textit{частично рекурсивной} (или \textit{вычислимой}), если существует алгоритм, который сможет вычислить её значение для любого допустимого аргумента за конечное количество времени.

Определение функции как \textit{частично рекурсивной} связано с понятием \textit{машин Тьюринга}, где машина Тьюринга, обладающая этим свойством, способна вычислить функцию для любого данного ввода, либо продолжать бесконечно, если функция не определена для этого ввода.

Функция называется \textit{тотально рекурсивной} или \textit{полностью вычислимой}, если она частично рекурсивная и определена для каждого ввода. То есть, машина Тьюринга, обрабатывающая такую функцию, гарантированно остановится для любого ввода.

Это является основой для определения понятия \textit{разрешимости} и \textit{полуразрешимости} проблем в теории вычислимости.

\section{Maszyna Turinga jako model procesów obliczalnych. }

\textit{Машина Тьюринга} -- это теоретическая модель вычислений, которая была предложена английским математиком Аланом Тьюрингом в 1936 году. Машина Тьюринга представляет собой простую, но мощную модель, способную вычислять любые задачи, которые можно описать алгоритмически.

Машина Тьюринга состоит из бесконечной ленты, разделенной на ячейки, и головки чтения/записи, которая может двигаться влево или вправо по ленте. Каждая ячейка может содержать один символ из конечного алфавита. Машина работает по определенным правилам, которые определяют, что она должна делать в зависимости от текущего состояния и символа, который она читает на ленте.

Говорят, что функция \textit{вычислима} на машине Тьюринга, если существует программа для этой машины, которая может вычислить функцию для любого допустимого ввода и остановится с правильным результатом.

Подобные простые машины могут казаться очень далекими от современных компьютеров, но каждый компьютер, с которым мы сталкиваемся в повседневной жизни, от мобильных телефонов до мощных суперкомпьютеров, может быть смоделирован машиной Тьюринга. Это приводит нас к утверждению, которое называется \textit{тезисом Чёрча-Тьюринга}, оно гласит, что любая функция, которая может быть вычислена физической машиной, может быть вычислена машиной Тьюринга. 


\section{Zagadnienia nierostrzygalne w kontekście obliczalności.}



В области теории вычислимости, \textit{неразрешимые задачи} – это такие задачи, которые невозможно решить с помощью какого-либо алгоритма. Другими словами, не существует машины Тьюринга, которая могла бы решить эти задачи для всех возможных входных данных.

Самый известный пример неразрешимой задачи -- это проблема остановки (halting problem). Эта задача состоит в определении, остановится ли определенная машина Тьюринга при работе с заданным вводом или будет работать бесконечно. Алан Тьюринг доказал, что эта проблема неразрешима в общем случае.

Неразрешимые проблемы не ограничиваются только проблемой остановки. Существует множество других неразрешимых проблем, многие из которых возникают в области теории чисел и формальных систем. Например, десятая проблема Гильберта, которая спрашивает о наличии общего алгоритма для определения, существуют ли целочисленные решения уравнения Диофанта. Эта проблема была признана неразрешимой Юрием Матиясевичем в 1970 году.
 