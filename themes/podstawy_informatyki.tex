\chapter{Podstawy informatyki}
\section{Problemy rekurencyjne i ich rozwiązywanie. }

\textbf{Рекурсия} в программировании это техника, при которой функция вызывает саму себя один или несколько раз. Важным моментом при использовании рекурсии является определение "базового случая", который не вызывает сам себя, чтобы предотвратить бесконечную рекурсию.

Рекурсия часто используется для решения задач, которые могут быть разбиты на более мелкие/простые подзадачи того же типа.

Например, факториал числа - это классическая рекурсивная задача. Факториал числа n (обозначается n!) — это произведение всех натуральных чисел от 1 до n включительно. Факториал можно определить рекурсивно, так как факториал n это произведение n на факториал n-1, а факториал 1 равен 1.

\begin{lstlisting}[language=Java]
int factorial(int n) {
if (n == 1) {
return 1;
} else {
return n * factorial(n - 1);
}
}
\end{lstlisting}

Важно отметить, что хотя рекурсия может быть мощным инструментом, она может также привести к проблемам с производительностью и переполнением стека, если рекурсивные вызовы становятся слишком глубокими. В некоторых случаях эти проблемы можно избежать с помощью техник, таких как мемоизация или преобразование рекурсии в итерацию.

\section{Pozycyjne systemy liczbowe i konwersje pomiędzy nimi. }

\textbf{Позиционные системы счисления} используются для представления чисел, причем важную роль играет позиция каждой цифры. Вот несколько распространенных систем счисления:

\begin{itemize}
\item \textbf{Десятичная система:} Использует 10 различных цифр (0-9). Это наиболее распространенная система счисления в повседневной жизни.

\item \textbf{Бинарная система:} Использует только две цифры: 0 и 1. Бинарная система широко используется в компьютерах, так как информация в них хранится в двоичной форме.

\item \textbf{Восьмеричная система:} Использует восемь цифр (0-7). Восьмеричные числа иногда используются в программировании.

\item \textbf{Шестнадцатеричная система:} Использует 16 цифр. Для представления чисел от 10 до 15 используются буквы A-F. Шестнадцатеричная система часто используется в программировании и информатике.
\end{itemize}

\textbf{Конвертация между различными системами счисления} может быть выполнена с использованием различных методов. Например, для конвертации числа из бинарной системы в десятичную, можно взять каждую цифру, умножить ее на два в степени, равной ее позиции (начиная с нуля справа), и сложить все эти значения. Обратный процесс можно использовать для конвертации из десятичной системы в бинарную.

Во многих языках программирования есть встроенные функции для выполнения таких преобразований. Например, в Java можно использовать Integer.parseInt() для преобразования бинарной строки в десятичное число, и Integer.toBinaryString() для преобразования десятичного числа в бинарную строку.

\section{Typ, zmienna, obiekt i zarządzanie pamięcią.}

\textbf{Тип:} В программировании тип данных определяет множество значений, которые может принимать переменная, а также набор операций, которые могут быть выполнены над этими значениями. Примеры типов данных включают целые числа (int), числа с плавающей запятой (float), булевы значения (boolean) и строки (String).

\textbf{Переменная:} Переменная это элемент программы, который может хранить значение определенного типа. Имя переменной используется для обращения к этому значению в программе.

\textbf{Объект:} В объектно-ориентированном программировании объект представляет собой экземпляр класса. Объекты могут содержать поля (переменные, связанные с объектом) и методы (функции, которые могут быть выполнены объектом).

\textbf{Управление памятью:} Управление памятью это аспект программирования, который относится к контролю над использованием памяти в компьютерной программе. В некоторых языках программирования, таких как C и C++, программистам приходится вручную выделять и освобождать память. Другие языки, такие как Java и Python, автоматически управляют памятью с помощью механизма под названием "сборка мусора", который автоматически освобождает память, которая больше не используется программой.

\section{Instrukcje sterujące przepływem programu. }

\textbf{Управляющие инструкции} в программировании используются для управления порядком, в котором выполняются инструкции. Они позволяют программам принимать решения и повторять действия. Вот некоторые основные типы управляющих инструкций:

\begin{itemize}
\item \textbf{Условные инструкции (if, switch):} Эти инструкции позволяют программе выбирать между различными путями выполнения на основе результатов проверки некоторых условий. Например, в Java оператор if может быть использован так:

\begin{lstlisting}[language=Java]
if (condition) {

} else {

}
\end{lstlisting}

\item \textbf{Циклы (for, while, do-while):} Циклы позволяют повторять блок кода несколько раз. Вот пример цикла for в Java:

\begin{lstlisting}[language=Java]
for (int i = 0; i < 10; i++) {
// loop 10
}
\end{lstlisting}

\item \textbf{Break и continue:} Эти инструкции используются для изменения нормального порядка выполнения циклов. Break прерывает цикл, а continue пропускает оставшуюся часть текущей итерации и переходит к следующей итерации.

\item \textbf{Return:} Используется для возвращения значения из функции и немедленного прекращения выполнения этой функции.
\end{itemize}

Эти управляющие инструкции обеспечивают гибкость и мощь, которые делают программирование полезным для решения широкого спектра задач.

\section{Kodowanie liczb ze znakiem w systemie U2, generowanie liczby ze znakiem przeciwnym, dodawanie i odejmowanie.}

\textbf{Дополнительный код (U2)} используется для представления отрицательных чисел в двоичной системе. В этом коде отрицательные числа кодируются как дополнение до двух их абсолютных значений. Для получения дополнительного кода к двум числа, можно инвертировать все биты (менять 0 на 1 и наоборот), а затем прибавить 1 к получившемуся числу.

Для примера, преобразуем -5 в двоичной системе в 8-битной форме:

\begin{itemize}
\item Запишем 5 в двоичной форме (7-битный код): 0000101
\item Инвертируем биты: 1111010
\item Прибавим 1: 1111011
\end{itemize}

Получившийся результат 1111011 — это представление числа -5 в 8-битной форме в системе U2.

Для выполнения \textbf{операций сложения и вычитания}, можно сложить числа как обычно, не обращая внимания на знаки. Если в результате сложения возникает перенос из старшего разряда, его следует отбросить. Если вычитаемое число больше уменьшаемого, результат должен взяться с обратным знаком.

Основной преимуществом кода U2 является его простота использования при выполнении арифметических операций, поскольку используются стандартные процедуры сложения и вычитания, а также удобство обработки знака.

