\chapter{BSK}
\section{Funkcje skrótu (mieszające) i ich zastosowania. }
Функции хэширования, или как они иногда называются, функции сокращения, являются важными инструментами в компьютерной науке и используются в различных областях, включая криптографию, поиск данных и обнаружение ошибок. Эти функции берут входные данные и преобразуют их в обычно более короткий, фиксированный размер выходных данных, называемый хешем. Хеш-функции должны быть детерминированными, что означает, что один и тот же вход всегда будет давать один и тот же выход.

Вот некоторые из применений хеш-функций:

\begin{enumerate}
\item \textbf{Структуры данных:} Хеш-функции используются в структурах данных, таких как хеш-таблицы, для быстрого доступа к данным.
\item \textbf{Криптография:} В криптографии хеш-функции используются для создания цифровых подписей, проверки целостности данных и сохранения паролей.
\item \textbf{Кэширование:} Хеш-функции могут использоваться для кэширования данных, когда нужно быстро проверить, содержится ли значение в наборе данных.
\item \textbf{Обнаружение ошибок:} Хеш-функции могут использоваться для обнаружения ошибок при передаче данных. Если хеш отправленных данных не соответствует хешу принятых данных, это указывает на ошибку в данных.
\end{enumerate}

\section{Atrybuty bezpieczeństwa informacji.}

Безопасность информации часто описывается с помощью трех основных атрибутов, известных как "Триада CIA" - Конфиденциальность, Целостность и Доступность.

\begin{itemize}
\item \textbf{Конфиденциальность (Confidentiality):} Этот атрибут гарантирует, что информация доступна только для тех лиц, которые имеют право ее просматривать. На практике конфиденциальность может быть обеспечена с помощью различных методов, включая использование паролей, шифрования и контроля доступа.

\item \textbf{Целостность (Integrity):} Целостность означает сохранность и точность информации. Этот атрибут гарантирует, что информация не была изменена без авторизации, что означает защиту от несанкционированного или случайного изменения. Целостность может быть обеспечена с помощью контрольных сумм, хэш-функций и цифровых подписей.

\item \textbf{Доступность (Availability):} Доступность означает, что информация и соответствующие ресурсы доступны тем лицам, которым они нужны, когда они нужны. Недоступность системы может быть результатом атаки или неполадок оборудования. Доступность обычно обеспечивается через резервное копирование, отказоустойчивость и обслуживание оборудования.
\end{itemize}

Важно отметить, что для обеспечения безопасности информации необходимо обеспечить все три атрибута. Например, система может быть конфиденциальной и доступной, но если информация в ней может быть изменена несанкционированно, система не может считаться безопасной.

\section{Modele dystrybucji kluczy kryptograficznych.}

Модели распределения ключей криптографии являются важным аспектом в обеспечении безопасного обмена информацией. Вот некоторые основные модели:

\begin{itemize}
\item \textbf{Симметричное шифрование (Private Key Cryptography):} В этой модели используется один и тот же ключ для шифрования и дешифрования сообщений. Проблема с этим подходом состоит в том, что безопасное распределение этого секретного ключа между обеими сторонами может быть сложной задачей, поскольку его необходимо передать через безопасный канал.

\item \textbf{Асимметричное шифрование (Public Key Cryptography):} В этой модели используется пара ключей: открытый ключ для шифрования и приватный ключ для дешифрования сообщений. Открытый ключ может быть свободно распространяем, а приватный ключ должен оставаться в тайне. Преимуществом этого подхода является то, что секретный ключ никогда не передается, что уменьшает риск его перехвата.

\item \textbf{Инфраструктура открытых ключей (Public Key Infrastructure, PKI):} Эта модель использует асимметричное шифрование, но также добавляет третью сторону, известную как сертификационный центр (CA), который выпускает цифровые сертификаты, подтверждающие владение парой ключей.

\item \textbf{Diffie-Hellman key exchange:} Этот протокол позволяет двум сторонам сгенерировать секретный ключ через незащищенный канал без передачи самого ключа. Это достигается за счет создания временных ключей, которые затем используются для создания общего секретного ключа.

\item \textbf{Key Distribution Center (KDC):} В этой модели существует третья доверенная сторона, которая хранит пары симметричных ключей для каждой стороны. Когда две стороны хотят общаться, они обращаются к KDC за симметричными ключами.
\end{itemize}

Выбор подходящей модели распределения ключей зависит от многих факторов, включая уровень безопасности, который необходим для конкретной задачи, доступность каналов связи и возможность использовать доверенные третьи стороны.

\section{Rodzaje zagrożeń oraz ochrona aplikacji sieciowych}

Приложения сети подвержены множеству угроз, и для обеспечения их безопасности необходимо учесть множество аспектов. Вот некоторые из основных типов угроз, а также способы их защиты:

\begin{itemize}
\item \textbf{Атаки на отказ в обслуживании (DoS и DDoS):} Эти атаки направлены на перегрузку системы или сети таким образом, что нормальные пользователи не могут получить доступ к сервису. Защита от таких атак может включать мониторинг трафика, ограничение количества запросов от одного источника или использование специализированных сервисов для защиты от DDoS.

\item \textbf{Прослушивание (Eavesdropping)/ Перехват (Interception):} Злоумышленники могут попытаться перехватить и прочитать незашифрованные данные, передаваемые через сеть. Для защиты можно использовать шифрование данных, транслируемых через сеть, а также использовать безопасные протоколы передачи данных, такие как HTTPS.

\item \textbf{Внедрение кода и SQL-инъекции:} Злоумышленники могут попытаться внедрить вредоносный код в приложение или использовать SQL-инъекции для доступа к базам данных. Защита включает в себя валидацию и очистку всех входных данных, использование параметризованных запросов и регулярное обновление и патчинг приложений и систем.

\item \textbf{Спуфинг и фишинг:} Атакующие могут попытаться подделать свою идентичность или идентичность доверенного веб-сайта, чтобы получить чувствительную информацию. Защита от этих атак включает в себя обучение пользователей, использование анти-фишинговых фильтров и сертификатов SSL.

\item \textbf{Взлом паролей:} Злоумышленники могут попытаться угадать пароли пользователей. Защита включает использование сильных паролей, ограничение числа попыток ввода пароля, использование двухфакторной аутентификации и хэширование паролей при хранении.
\end{itemize}

Обеспечение безопасности сетевого приложения - это непрерывный процесс, который включает мониторинг угроз, реагирование на инциденты безопасности, обновление и улучшение мер безопасности по мере развития новых угроз.

\section{Charakterystyka kryptografii symetrycznej oraz asymetrycznej.}


\textbf{Криптография с симметричным ключом}, также известная как секретная ключевая криптография, использует один и тот же ключ для шифрования и дешифрования данных. Это значит, что обе стороны (отправитель и получатель) должны знать ключ, чтобы безопасно обмениваться информацией. Примерами симметричного шифрования являются DES, 3DES, AES и Blowfish. Преимуществом является скорость работы таких алгоритмов. Однако есть проблема с безопасной передачей ключа между сторонами.


\begin{lstlisting}[language=Java]
    Cipher cipher = Cipher.getInstance("AES");
    SecretKeySpec secretKey = new SecretKeySpec(key, "AES");
    cipher.init(Cipher.ENCRYPT_MODE, secretKey);
    byte[] encryptedText = cipher.doFinal(plainText.getBytes());
    \end{lstlisting}

\textbf{Криптография с асимметричным ключом}, также известная как криптография с открытым ключом, использует пару ключей: открытый ключ для шифрования и закрытый ключ для дешифрования. Это значит, что ключ, используемый для шифрования данных, отличается от ключа, используемого для их дешифрования. Примеры асимметричного шифрования включают RSA, DSA и ECC. Преимущество состоит в том, что открытый ключ можно свободно распространять, не беспокоясь о его перехвате. Однако такие алгоритмы работают медленнее по сравнению с симметричными.

\begin{lstlisting}[language=Java]
    KeyPairGenerator keyPairGenerator = KeyPairGenerator.getInstance("RSA");
    keyPairGenerator.initialize(2048);
    KeyPair keyPair = keyPairGenerator.generateKeyPair();
    
    Cipher cipher = Cipher.getInstance("RSA/ECB/PKCS1Padding");
    cipher.init(Cipher.ENCRYPT_MODE, keyPair.getPublic());
    
    byte[] encryptedText = cipher.doFinal(plainText.getBytes());
    \end{lstlisting}

    Важно отметить, что на практике часто используется комбинация обоих подходов. Например, в протоколе SSL/TLS сначала используется асимметричное шифрование для обмена секретным ключом, который затем используется для симметричного шифрования сеанса связи.






    
