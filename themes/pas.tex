\chapter{PAS}
\section{Mechanizm sesji w zarządzaniu stanem aplikacji sieciowej.}

Веб-приложения в основном являются "без сохранения состояния" (stateless), что означает, что каждый запрос обрабатывается независимо от других. Это может создать проблемы, когда необходимо сохранить информацию о состоянии пользователя между несколькими запросами. Для решения этой проблемы веб-приложения часто используют механизм сессий.

Сессия - это способ сохранения информации (например, идентификатор пользователя или данные корзины покупок) на сервере между несколькими запросами. Когда пользователь делает первый запрос, сервер создает уникальный идентификатор сессии и отправляет его обратно в браузер в виде cookie. Затем при каждом последующем запросе браузер отправляет этот идентификатор сессии обратно на сервер, позволяя серверу "восстановить" состояние сессии.

Сессии важны для поддержания "состояния" между множеством запросов и являются основой для функциональности, такой как аутентификация пользователя, сохранение содержимого корзины покупок, отслеживание предпочтений пользователя и так далее.


\section{Mechanizm gniazd – pojęcie, sposób realizacji i zastosowanie }

Сокет - это абстракция, используемая для отправки и получения данных между процессами. Процессы могут находиться на одном компьютере или на разных компьютерах, соединенных через сеть.

\subsection*{Способ реализации}

Сокеты обычно реализуются в операционной системе как системные вызовы. В Unix-подобных операционных системах, например, предоставляются системные вызовы, такие как socket (для создания нового сокета), connect (для установления соединения с другим сокетом), и send и receive (для отправки и получения данных).

\subsection*{Застосування}

Сокеты используются для обмена данными между процессами в сети, и они являются основой для многих сетевых протоколов и технологий, таких как HTTP, FTP, SSH и так далее.

Например, когда вы открываете веб-страницу в браузере, браузер создает сокет и устанавливает соединение с веб-сервером, а затем использует этот сокет для отправки HTTP-запроса на сервер и получения HTTP-ответа от сервера.


\section{Metody obsługi wielu klientów równolegle w aplikacjach sieciowych.}


\subsection*{Многопоточность}

Многопоточность - это метод, при котором каждому клиенту назначается отдельный поток в рамках одного процесса. Это позволяет приложению обслуживать множество клиентов одновременно, каждый в своем потоке. Этот метод может быть эффективным, но требует аккуратной работы с потоками, включая синхронизацию и управление состоянием.

\subsection*{Многопроцессность}

Многопроцессность подразумевает использование отдельного процесса для каждого клиента. Это дает каждому клиенту свой собственный контекст выполнения и пространство памяти, что может упростить управление состоянием, но также может быть более ресурсоемким, чем многопоточность.

\subsection*{Асинхронная обработка}

Асинхронная обработка предполагает использование одного или нескольких потоков или процессов, которые могут обрабатывать множество клиентов одновременно, используя асинхронные операции ввода-вывода. Вместо блокировки и ожидания завершения операции ввода-вывода, асинхронное приложение будет продолжать обслуживание других клиентов. Это может быть очень эффективным, особенно для приложений с высокой загрузкой, но также требует от разработчика более сложной модели программирования.


\section{Pocztowe protokoły warstwy aplikacji. }

\subsection*{SMTP (Simple Mail Transfer Protocol)}

SMTP - это протокол, используемый для отправки электронной почты от отправителя к получателю и между серверами почты. Он работает над протоколом TCP/IP и использует порт 25. SMTP поддерживает только отправку текстовых сообщений.

\subsection*{POP3 (Post Office Protocol version 3)}

POP3 - это протокол, используемый клиентами электронной почты для получения сообщений с почтового сервера. Сообщения скачиваются на локальное устройство и обычно удаляются с сервера. Он работает над протоколом TCP/IP и использует порт 110.

\subsection*{IMAP (Internet Message Access Protocol)}

IMAP - это альтернатива POP3, которая предоставляет более сложные функции. В отличие от POP3, IMAP позволяет клиентам электронной почты сохранять сообщения на сервере, организовывать их в папки и синхронизировать структуру папок между несколькими устройствами. Он работает над протоколом TCP/IP и использует порт 143.


\section{Porównanie HTTP i WebSocket. }

\subsection*{HTTP}

HTTP (HyperText Transfer Protocol) - это протокол, широко используемый для обмена данными в веб-приложениях. HTTP основан на модели "запрос-ответ", где клиент отправляет запрос на сервер, а сервер затем возвращает ответ. 

HTTP не сохраняет состояние между различными запросами, что делает его "без сохранения состояния" (stateless). Кроме того, в HTTP только клиент может инициировать обмен данными, отправляя запрос на сервер.

\subsection*{WebSocket}

WebSocket - это протокол, позволяющий двустороннее взаимодействие между клиентом и сервером в реальном времени. В отличие от HTTP, WebSocket позволяет серверу инициировать передачу данных, а также поддерживает постоянное соединение, что обеспечивает более быструю обработку сообщений, поскольку не требуется переустановка соединения для каждого обмена данными.

WebSocket идеально подходит для приложений, которым требуется быстрое и постоянное взаимодействие между клиентом и сервером, таких как игры в реальном времени, чаты и применения в реальном времени.

