\chapter{ai}

\section{Metody uczenia maszynowego.}
Методы машинного обучения обычно делятся на три основные категории: обучение с учителем, обучение без учителя и обучение с подкреплением.

Обучение с учителем (Supervised Learning):
\begin{itemize}
\item В этом подходе модель обучается на основе обучающего набора данных, который содержит входные данные и соответствующие им ожидаемые выходные данные (метки классов или значения).
\item Задача модели состоит в том, чтобы научиться прогнозировать выходные данные на основе входных данных.
\item Примеры алгоритмов обучения с учителем включают линейную и логистическую регрессию, машины опорных векторов (SVM), и нейронные сети.
\end{itemize}

Обучение без учителя (Unsupervised Learning):
\begin{itemize}
\item Здесь модель обучается на основе набора данных, который содержит только входные данные, и нет конкретных ожидаемых выходных данных.
\item Задача модели состоит в том, чтобы найти скрытые структуры или закономерности в данных, такие как группы, паттерны или правила.
\item Примеры алгоритмов обучения без учителя включают K-средних, иерархическую кластеризацию и методы снижения размерности, такие как анализ основных компонентов (PCA).
\end{itemize}

Обучение с подкреплением (Reinforcement Learning):
\begin{itemize}
\item В этом подходе агент (модель) обучается, взаимодействуя со средой. Агент получает награды или штрафы (подкрепления) на основе своих действий, и цель состоит в том, чтобы максимизировать суммарную награду.
\item Обучение с подкреплением часто используется в областях, где требуется обучение последовательности действий, таких как игры, навигация и управление роботами.
\end{itemize}

Важно отметить, что это обобщения и многие алгоритмы машинного обучения могут быть классифицированы в более чем одну категорию.
Кроме того, существуют и другие типы машинного обучения, такие как полу-надежное обучение (semi-supervised learning) и обучение с активным использованием примеров (active learning), которые являются гибридами этих основных типов.


\section{Budowa sieci neuronowych}

Нейронная сеть состоит из нескольких слоёв, каждый из которых содержит набор нейронов. Каждый нейрон в свою очередь связан с нейронами в следующем слое через весовые коэффициенты. 

\subsection*{Нейрон}

Нейрон — это базовая единица в нейронной сети. Это функция, которая принимает вектор входных данных $x = (x_1, x_2, \dots, x_n)$ и весовых коэффициентов $w = (w_1, w_2, \dots, w_n)$, и возвращает выходное значение $y$. Математически это можно записать так:

\begin{equation}
y = f\left(\sum_{i=1}^{n} w_i x_i + b\right)
\end{equation}

где $b$ - это смещение (bias), а $f$ - функция активации, которая может быть различной (например, сигмоидальной, ReLU, гиперболического тангенса и т.д.).

\subsection*{Слои}

В нейронной сети различают следующие основные типы слоёв:

\begin{itemize}
\item Входной слой: этот слой получает входные данные.
\item Скрытые слои: эти слои обрабатывают входные данные. В них происходит большая часть вычислений нейронной сети.
\item Выходной слой: этот слой возвращает окончательный результат работы нейронной сети.
\end{itemize}

Веса между нейронами различных слоев обычно оптимизируются с помощью метода обратного распространения ошибки (backpropagation), чтобы минимизировать ошибку между предсказанными и реальными значениями на тренировочных данных.
