\chapter{Systemy operacyjne}
\section{Wielowarstwowa organizacja systemów komputerowych.}

\section{Многоуровневая организация компьютерных систем}

\subsection{Преимущества}

Преимущества многоуровневой организации включают следующее:

\begin{itemize}
\item \textbf{Модульность}: Слои можно разрабатывать и обновлять независимо друг от друга, что упрощает процесс разработки и обслуживания системы.
\item \textbf{Переносимость}: При должном проектировании функции и услуги, предоставляемые одним слоем, могут быть заменены без влияния на другие слои.
\item \textbf{Абстракция}: Каждый слой может сосредоточиться на выполнении своих специфических задач, не беспокоясь о деталях выполнения задач на других уровнях.
\end{itemize}

\subsection{Примеры}

Примеры многоуровневой организации включают модель взаимодействия открытых систем (OSI), которая используется в сетевой коммуникации, и архитектуру систем на основе микросервисов, в которой различные функции приложения разделяются на независимые службы, которые могут быть разработаны, развернуты и масштабированы независимо друг от друга.

\section{System operacyjny – charakterystyka, zadania, klasyfikacja. }

\textit{Операционная система} (ОС) - это системное программное обеспечение, которое управляет аппаратными ресурсами компьютера и предоставляет услуги для выполнения прикладного программного обеспечения. Она служит в качестве интерфейса между пользователем и аппаратными ресурсами системы.

\subsection{Характеристики}

Операционные системы могут быть характеризованы следующими свойствами:

\begin{itemize}
\item \textbf{Многозадачность}: Поддержка выполнения нескольких процессов или потоков в рамках одной системы.
\item \textbf{Многопользовательский режим}: Позволяет нескольким пользователям одновременно использовать ресурсы системы.
\item \textbf{Управление памятью}: Отслеживает использование памяти и координирует доступ к ней.
\item \textbf{Управление процессами}: Управляет выполнением процессов и распределением ресурсов процессора.
\item \textbf{Управление устройствами}: Координирует доступ и использование аппаратных ресурсов, таких как диски, принтеры и интерфейсы сети.
\end{itemize}

\subsection{Задачи}

Основные задачи операционной системы включают:

\begin{itemize}
\item \textbf{Управление ресурсами}: Координация и управление доступом к общим ресурсам, таким как процессор, память и устройства ввода-вывода.
\item \textbf{Обеспечение безопасности и доступа}: Защита данных и системных ресурсов от несанкционированного доступа и обеспечение контроля доступа.
\item \textbf{Обеспечение удобного интерфейса}: Предоставление удобного и эффективного интерфейса для пользователей и прикладных программ.
\end{itemize}

\subsection{Классификация}

Операционные системы можно классифицировать по различным критериям, включая:

\begin{itemize}
\item \textbf{Тип системы}: Настольные (например, Windows, macOS, Linux), мобильные (например, Android, iOS), серверные (например, Windows Server, Unix), встроенные и т.д.
\item \textbf{Архитектура}: Монолитная, микроядерная, многоядерная и т.д.
\end{itemize}



\section{Procesy i wątki – charakterystyka i problemy}
\subsection{Процессы}

\textit{Процесс} - это экземпляр программы во время выполнения, включающий в себя текущее состояние программы и всю информацию, необходимую для ее выполнения, такую как значения регистров процессора, системные ресурсы и область памяти.

\subsection{Потоки}

\textit{Поток} - это наименьшая единица обработки, которую можно запланировать и управлять с помощью операционной системы. Поток представляет собой подмножество процесса и делится на область памяти и ресурсы с другими потоками в том же процессе. Потоки могут выполняться параллельно, что позволяет более эффективно использовать многоядерные процессоры.

\subsection{Проблемы}

\begin{itemize}
\item \textbf{Взаимная блокировка}: Это состояние, при котором два или более процессов или потоков вечно ждут друг друга, чтобы освободить ресурсы. Это может привести к полному останову работы системы.
\item \textbf{Гонка данных}: Это состояние, при котором два или более потока неожиданно взаимодействуют при общем доступе к ресурсу, что приводит к некорректным или непредсказуемым результатам.
\item \textbf{Планирование и контекстные переключения}: Планирование процессов и потоков и переключение контекста между ними требуют значительных затрат системных ресурсов.
\end{itemize}

Многие из этих проблем могут быть смягчены с помощью техник синхронизации, таких как мьютексы, семафоры и мониторы.


\section{Zarządzanie pamięcią operacyjną w systemie operacyjnym.}

\textit{Управление оперативной памятью} - это одна из основных функций операционной системы. Она включает в себя управление физической и виртуальной памятью, выделение и освобождение памяти для процессов, а также защиту и изоляцию адресных пространств.

\subsection{Управление физической памятью}

Операционная система отслеживает доступную физическую память и отвечает за ее выделение процессам, когда они ее запрашивают. Операционная система также отвечает за освобождение памяти, когда процессы завершаются или когда они больше не нуждаются в выделенных ресурсах.

\subsection{Управление виртуальной памятью}

Операционная система может предоставлять процессам виртуальное адресное пространство, которое может быть больше, чем реально доступная физическая память. Это достигается путем использования подкачки и страничного обмена, при которых неиспользуемые страницы памяти могут быть временно перемещены на диск.

\subsection{Защита памяти}

Операционная система также отвечает за защиту памяти, гарантируя, что процессы не могут читать или записывать в адресное пространство других процессов, если только это явно не разрешено. Это обеспечивает безопасность и стабильность системы.

\subsection{Сегментация и страничное разделение}

Операционная система может использовать сегментацию или страничное разделение для управления памятью. Сегментация заключается в разделении адресного пространства процесса на части, или сегменты, разного размера. Страничное разделение разделяет память на блоки фиксированного размера, или страницы.


\section{Organizacja systemu plików i pamięci zewnętrznej.}
\subsection{Система файлов}

\textit{Система файлов} - это метод организации и хранения информации на внешнем устройстве хранения, таком как жесткий диск, SSD или съемный накопитель. Она определяет, как файлы и каталоги структурированы и как они связаны между собой. Существуют различные типы файловых систем, включая FAT32, NTFS, HFS+, Ext4 и другие, каждая из которых имеет свои особенности и применение.

\subsection{Внешняя память}

\textit{Внешняя память} относится к устройствам хранения данных, которые не используются для выполнения программ, но используются для долгосрочного хранения данных. Это включает в себя различные типы устройств, такие как жесткие диски, твердотельные накопители (SSD), оптические диски и съемные накопители.

\subsection{Организация}

Операционная система использует систему файлов для организации данных на внешней памяти. Файлы обычно группируются в каталоги (также называемые "папками") для удобства навигации и организации. Файлы и каталоги могут иметь атрибуты, такие как права доступа, дата создания и владелец, которые контролируют, как файлы могут быть прочитаны, записаны и изменены.

\subsection{Управление памятью}

Операционная система также отвечает за управление внешней памятью. Это включает в себя отслеживание доступного пространства, выделение пространства для файлов, освобождение пространства, когда файлы удаляются, и многое другое. Операционная система также может использовать техники, такие как фрагментация и дефрагментация, чтобы оптимизировать использование пространства и производительность чтения/записи.
