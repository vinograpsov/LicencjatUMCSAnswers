\chapter{Statystyka}

\section{Основные характеристики описательной и математической статистики}

\textit{Статистика} – это отрасль математики, которая занимается сбором, анализом, интерпретацией, представлением и организацией данных. Она делится на две основные ветви: описательную статистику и инференциальную (или математическую) статистику.

\subsection{Описательная статистика}

Описательная статистика, как следует из названия, занимается описанием или суммированием набора данных. Она включает в себя расчет различных статистических показателей, таких как среднее значение, медиана, мода, стандартное отклонение, дисперсия и коэффициенты корреляции. Описательная статистика также может включать в себя создание графических представлений данных, таких как гистограммы, ящики с усами и диаграммы рассеяния.

\subsection{Математическая статистика}

Математическая (или инференциальная) статистика использует математические методы для обобщения данных и делает выводы или предсказания на основе набора данных. Инференциальная статистика используется для оценки параметров, проверки статистических гипотез, построения доверительных интервалов и прогнозирования будущих результатов на основе выборки данных.

Важно отметить, что описательная статистика и инференциальная статистика не являются взаимоисключающими и часто используются вместе в статистическом анализе.

