\chapter{Sieci}
\section{Protokoły TCP i UDP – porównanie i zastosowanie.}


\textit{TCP} (Transmission Control Protocol) и \textit{UDP} (User Datagram Protocol) – это два основных протокола транспортного уровня, используемых в интернете. Они обеспечивают различные методы доставки данных от одного устройства к другому.

\subsection{TCP}

TCP является надежным протоколом, который гарантирует доставку пакетов данных. Он предоставляет функции, такие как контроль потока, контроль перегрузки и повторная передача потерянных пакетов. Это делает TCP идеальным для приложений, которым требуется гарантированная доставка данных, таких как веб-браузеры, электронная почта и файловые передачи.

\subsection{UDP}

В отличие от TCP, UDP является ненадежным протоколом, который не гарантирует доставку пакетов данных. Он просто отправляет пакеты без установления соединения или проверки, что пакеты были доставлены. Это делает UDP более быстрым и эффективным для приложений, которым не требуется гарантированная доставка, таких как видео и аудио потоки, игры и некоторые виды VoIP (Voice over IP).

\subsection{Сравнение}

Вот несколько ключевых различий между TCP и UDP:

\begin{itemize}
\item \textbf{Надежность}: TCP обеспечивает надежность за счет повторной передачи потерянных пакетов и проверки ошибок, в то время как UDP не предоставляет такой гарантии.
\item \textbf{Скорость}: UDP обычно быстрее, чем TCP, поскольку он не заботится о повторной передаче потерянных пакетов.
\item \textbf{Соединения}: TCP использует принцип соединений, то есть перед передачей данных устанавливается соединение. UDP не требует установления соединения перед отправкой данных.
\item \textbf{Порядок пакетов}: TCP гарантирует, что пакеты будут доставлены в том порядке, в котором они были отправлены. UDP не предоставляет такую гарантию.
\end{itemize}

В целом, выбор между TCP и UDP зависит от приложения и требуемых свойств передачи данных.

\section{Adresowanie w warstwie Internetu modelu TCP/IP. }

В модели TCP/IP \textit{адресация в интернет-слое} относится к способу, которым устройства идентифицируются и локализуются в сети. Это в основном происходит через протоколы IP-адресации, а именно IPv4 и IPv6.

\subsection{IPv4}

IPv4 (Internet Protocol version 4) является наиболее широко использованным протоколом IP. IPv4-адрес состоит из 32 битов и обычно записывается в виде четырех десятичных чисел, разделенных точками, каждое из которых представляет восьмибитное число от 0 до 255. Например, 192.168.0.1.

\subsection{IPv6}

IPv6 (Internet Protocol version 6) был разработан как решение проблемы исчерпания адресов IPv4. IPv6-адрес состоит из 128 битов и записывается как восемь групп из четырех шестнадцатеричных чисел, разделенных двоеточиями. Например, 2001:0db8:85a3:0000:0000:8a2e:0370:7334.

\subsection{Сети и подсети}

IP-адреса обычно включают идентификатор сети и идентификатор устройства внутри этой сети (хоста). Это позволяет сетевым устройствам маршрутизировать пакеты на правильную сеть и к правильному устройству внутри этой сети. Маска подсети используется для разделения адреса на сетевую и хостовую части.

Важно отметить, что хотя IPv4 и IPv6 используют разные форматы адресов, оба они работают по тому же основному принципу, и оба могут быть использованы в интернет-слое модели TCP/IP.


\section{Porównanie zadań przełącznika (switcha) i routera.}

\textit{Переключатели (switches)} и \textit{маршрутизаторы (routers)} являются ключевыми компонентами сетевой инфраструктуры, но они выполняют разные функции и работают на разных уровнях модели OSI.

\subsection{Переключатель (Switch)}

Переключатель – это устройство, работающее на уровне данных (2-й уровень) модели OSI. Он используется для соединения устройств внутри одной локальной сети (LAN). Переключатели могут "изучить" MAC-адреса устройств, подключенных к ним, и направлять трафик непосредственно от отправителя к получателю в пределах одной сети. Это повышает эффективность сети, уменьшая коллизии и увеличивая пропускную способность.

\subsection{Маршрутизатор (Router)}

Маршрутизатор – это устройство, работающее на сетевом уровне (3-й уровень) модели OSI. Он соединяет две или более сетей вместе и управляет трафиком между ними. Маршрутизаторы используют IP-адреса для определения лучшего пути для передачи пакетов между сетями. Они обеспечивают связь между локальными сетями и глобальной сетью (например, Интернетом), а также могут обрабатывать такие задачи, как преобразование адресов (NAT), фильтрация трафика и безопасность сети.

\subsection{Сравнение}

В общем, можно сказать, что переключатели используются для управления трафиком внутри одной сети, а маршрутизаторы – для управления трафиком между разными сетями. Оба они играют важную роль в современной сетевой инфраструктуре.


\section{Porównanie modelu OSI i TCP/IP.}

\textit{Модель OSI} (Open Systems Interconnection) и \textit{модель TCP/IP} (Transmission Control Protocol/Internet Protocol) – это две концептуальные модели, которые используются для описания, как сетевые протоколы взаимодействуют и коммуницируют друг с другом. 

\subsection{Модель OSI}

Модель OSI состоит из семи слоев: Физический, Канальный, Сетевой, Транспортный, Сеансовый, Представления и Прикладной. Эта модель была разработана Международной организацией по стандартизации (ISO) и представляет собой идеализированное представление о том, как должны взаимодействовать сетевые протоколы.

\subsection{Модель TCP/IP}

Модель TCP/IP, также известная как Интернет-модель, состоит из четырех слоев: Сетевого интерфейса, Интернета, Транспорта и Приложения. Эта модель была разработана в рамках проекта DARPA и является основой современного Интернета.

\subsection{Сравнение}

Вот несколько ключевых отличий между моделями OSI и TCP/IP:

\begin{itemize}
\item \textbf{Количество слоев}: Модель OSI состоит из семи слоев, в то время как модель TCP/IP имеет только четыре.
\item \textbf{Абстракция vs реальность}: Модель OSI часто рассматривается как идеальная, но искусственная модель, в то время как модель TCP/IP базируется на реальных протоколах, используемых в Интернете.
\item \textbf{Универсальность}: Модель OSI стремится быть универсальной моделью для всех видов сетевых протоколов, в то время как модель TCP/IP была разработана специально для семейства протоколов Интернета.
\end{itemize}

Обе модели играют важную роль в понимании и описании сетевых взаимодействий и протоколов.



\section{Mechanizm enkapsulacji w modelu OSI.}

\textit{Инкапсуляция} является ключевым процессом в модели OSI, который позволяет данным перемещаться от верхних слоев модели к нижним и обратно. На каждом слое информация, которую нужно передать, оборачивается в капсулу с добавлением заголовка (и, возможно, трейлера), содержащего специфическую для этого слоя информацию. 

При передаче данных от источника к получателю процесс инкапсуляции происходит следующим образом:

\begin{enumerate}
\item Данные начинают свое путешествие на прикладном слое (седьмой слой) модели OSI на стороне отправителя.
\item На каждом следующем слое к данным добавляется заголовок (и иногда трейлер), содержащий информацию, специфичную для этого слоя. Этот процесс продолжается до тех пор, пока данные не достигнут физического слоя (первого слоя) и не будут переданы по сети.
\item По прибытии данных к получателю процесс инкапсуляции происходит в обратном порядке (это называется декапсуляцией). На каждом слое заголовки (и трейлеры), добавленные на стороне отправителя, удаляются, и данные передаются на следующий слой вверх по стеку, пока не достигнут прикладного слоя.
\end{enumerate}

Этот процесс позволяет каждому слою модели OSI фокусироваться на своей конкретной задаче и обрабатывать специфическую для слоя информацию, не заботясь о деталях, специфических для других слоев.

