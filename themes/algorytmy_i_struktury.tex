\chapter{Algorytmy i struktury danych}
\section{Paradygmat i przykłady programowania generycznego (rodzajowego). }


\subsection*{Парадигма программирования с использованием обобщений}

Программирование с использованием обобщений — это парадигма программирования, которая позволяет писать код, который может работать с различными типами данных. Это делает код более переиспользуемым и безопасным в работе с типами.

\subsection*{Пример использования обобщений}

\begin{lstlisting}[language=Java]
public class Box<T> {
    private T content;

    public void setContent(T content) {
        this.content = content;
    }

    public T getContent() {
        return this.content;
    }
}
\end{lstlisting}

В этом примере, `Box` — это обобщённый класс, который может хранить объекты любого типа. `T` — это типовой параметр, который будет заменён на реальный тип при использовании класса `Box`.

\begin{lstlisting}[language=Java]
Box<String> stringBox = new Box<>();
stringBox.setContent("Hello, world!");
System.out.println(stringBox.getContent()); // Output: Hello, world!

Box<Integer> integerBox = new Box<>();
integerBox.setContent(123);
System.out.println(integerBox.getContent()); // Output: 123
\end{lstlisting}

Здесь `Box<String>` и `Box<Integer>` — это два разных типа, созданных на основе обобщённого класса `Box`.

\section{Algorytmy sortowania. }

Сортировка - это фундаментальная задача в информатике, которая заключается в упорядочении набора данных на основе определенного критерия. Важно отметить, что эффективность алгоритма сортировки может сильно варьироваться в зависимости от его характеристик и типа входных данных.

\subsection*{Bubble Sort}

Сортировка пузырьком работает, перебирая список от начала до конца, сравнивая пары соседних элементов и меняя их местами, если они находятся в неправильном порядке. Этот процесс повторяется, пока весь список не будет отсортирован.


\begin{lstlisting}
    public static void bubbleSort(int[] arr) {
        int n = arr.length;
        for (int i = 0; i < n-1; i++)
            for (int j = 0; j < n-i-1; j++)
                if (arr[j] > arr[j+1]) {
                    // swap arr[j+1] and arr[j]
                    int temp = arr[j];
                    arr[j] = arr[j+1];
                    arr[j+1] = temp;
                }
    }
    \end{lstlisting}

\subsection*{Insertion Sort}

Алгоритм сортировки вставками работает, делая в каждой итерации список частично отсортированным. В каждом шаге алгоритма выбирается один из элементов и вставляется на правильное место в уже отсортированной части списка.


\begin{lstlisting}
    public static void insertionSort(int[] arr) {
        int n = arr.length;
        for (int i = 1; i < n; ++i) {
            int key = arr[i];
            int j = i - 1;
    
            while (j >= 0 && arr[j] > key) {
                arr[j + 1] = arr[j];
                j = j - 1;
            }
            arr[j + 1] = key;
        }
    }
    \end{lstlisting}

\subsection*{Selection Sort}

Сортировка выбором работает, повторно находя минимальный (или максимальный, в зависимости от порядка сортировки) элемент из нераспределённой части списка и помещая его в конец отсортированной части. Этот процесс продолжается, пока все элементы не будут отсортированы.

\begin{lstlisting}
    public static void selectionSort(int[] arr) {
        int n = arr.length;
    
        for (int i = 0; i < n-1; i++) {
            int min_idx = i;
            for (int j = i+1; j < n; j++)
                if (arr[j] < arr[min_idx])
                    min_idx = j;
    
            // Swap the found minimum element with the first element
            int temp = arr[min_idx];
            arr[min_idx] = arr[i];
            arr[i] = temp;
        }
    }
    \end{lstlisting}

\subsection*{Quick Sort}

Быстрая сортировка — это эффективный алгоритм сортировки, который работает по принципу "разделяй и властвуй". В этом алгоритме выбирается "опорный" элемент, и остальные элементы распределяются так, чтобы все элементы, меньшие опорного, шли до него, а все элементы, большие опорного, шли после.

\begin{lstlisting}
    public static void quickSort(int[] arr, int begin, int end) {
        if (begin < end) {
            int partitionIndex = partition(arr, begin, end);
    
            quickSort(arr, begin, partitionIndex-1);
            quickSort(arr, partitionIndex+1, end);
        }
    }
    
    private static int partition(int[] arr, int begin, int end) {
        int pivot = arr[end];
        int i = (begin-1);
    
        for (int j = begin; j < end; j++) {
            if (arr[j] <= pivot) {
                i++;
    
                int swapTemp = arr[i];
                arr[i] = arr[j];
                arr[j] = swapTemp;
            }
        }
    
        int swapTemp = arr[i+1];
        arr[i+1] = arr[end];
        arr[end] = swapTemp;
    
        return i+1;
    }
    \end{lstlisting}

\subsection*{Merge Sort}

Сортировка слиянием также работает по принципу "разделяй и властвуй". В этом алгоритме список разбивается на две половины, каждая из которых сортируется отдельно, а затем две отсортированные половины объединяются в один список.


\begin{lstlisting}
    public static void mergeSort(int[] arr, int begin, int end) {
        if (begin < end) {
            int mid = (begin + end) / 2;
    
            mergeSort(arr, begin, mid);
            mergeSort(arr , mid+1, end);
    
            merge(arr, begin, mid, end);
        }
    }
    
    private static void merge(int[] arr, int begin, int mid, int end) {
        // merging code here...
    }
    \end{lstlisting}

\section{Strategia „dziel i zwyciężaj” budowania algorytmów.}


"Разделяй и властвуй" — это методология, используемая в проектировании алгоритмов. Она основана на рекурсивном разделении задачи на подзадачи, которые являются меньшими версиями исходной задачи. Эти подзадачи решаются индивидуально, а затем их решения комбинируются для получения ответа на исходную задачу. Стратегия состоит из следующих основных шагов:

"Разделяй и властвуй" - это методология, используемая в проектировании алгоритмов. Она включает рекурсивное разбиение задачи на подзадачи, решение этих подзадач и комбинирование их решений для получения решения исходной задачи. Стратегия состоит из следующих шагов:

\begin{enumerate}
\item \textbf{Разделение:} Задача разбивается на подзадачи.
\item \textbf{Властвование:} Подзадачи решаются. Если подзадачи все еще слишком большие, они могут быть разбиты еще больше.
\item \textbf{Комбинирование:} Решения подзадач объединяются, чтобы получить окончательное решение.
\end{enumerate}

Примеры алгоритмов, которые используют эту стратегию, включают быструю сортировку, сортировку слиянием, алгоритмы для нахождения минимального остовного дерева и алгоритмы быстрого возведения в степень.

\begin{lstlisting}
public static double power(double a, int n) {
    if (n == 0) {
        return 1;
    } else if (n % 2 == 0) {
        double temp = power(a, n / 2);
        return temp * temp;
    } else {
        return a * power(a, n - 1);
    }
}
\end{lstlisting}

В этом примере, функция `power` рекурсивно вызывает себя, уменьшая степень `n` с каждым вызовом. Это уменьшение степени и есть "разделение" проблемы. "Властвование" происходит при выполнении умножения в каждом вызове функции, а "комбинирование" решений подзадач происходит автоматически, так как каждый рекурсивный вызов возвращает результат, который умножается на результат другого вызова (или на основание `a`, в случае нечетной степени).

\section{Algorytmy typu zachłannego.}


Жадные алгоритмы — это класс алгоритмов оптимизации, которые решают задачу пошаговым выбором локально оптимального решения в каждый момент времени с надеждой, что набор этих локальных решений приведет к глобальному оптимальному решению.

Основная идея жадного алгоритма заключается в принятии решения, которое кажется наиболее оптимальным в данный момент, и полном игнорировании последствий этих решений для последующих шагов.

\section{Algorytmy z nawrotami.}

Алгоритмы с возвратами, или "backtracking" алгоритмы, представляют собой стратегию решения проблем, которая включает обход дерева возможных решений для задачи. Этот обход может быть описан следующими шагами:

\begin{enumerate}
\item Выбор опции из набора возможных решений.
\item Проверка, приведет ли выбор этой опции к решению проблемы. Если да, то решение найдено.
\item Если выбор не привел к решению, возвращение к шагу 1 и выбор следующей опции.
\item Если все опции были проверены и ни одна из них не привела к решению, возвращение на шаг назад и выбор следующей опции на этом уровне.
\end{enumerate}

Этот подход используется в решении многих задач, включая задачу о восьми ферзях, задачу о гамильтоновом цикле, задачу о раскраске графа и решение головоломок, таких как "Судоку".


\section{Grafy, drzewa, kopce – charakterystyka i przykłady zastosowania.}

\subsection*{Графы}

Граф - это структура данных, которая состоит из вершин, некоторые из которых соединены ребрами. Графы могут быть направленными (где каждое ребро направлено от одной вершины к другой) или ненаправленными. 

Примеры использования графов включают моделирование сетей (таких как интернет или социальные сети), анализ связности и пути в географии или транспортной сети и многое другое.

\subsection*{Деревья}

Дерево - это специальный тип графа, который не содержит циклов. Дерево имеет корневую вершину и каждая вершина имеет ноль или более дочерних вершин. 

Деревья широко используются в компьютерных науках. Например, деревья используются для представления иерархических отношений (как в системах управления файлами), для обработки и анализа данных (в бинарных деревьях поиска, синтаксических деревьях в компиляторах), в машинном обучении и т.д.

\subsection*{Кучи}

Куча - это специальный вид бинарного дерева (обычно полного бинарного дерева), где каждая вершина имеет значение, которое больше или равно (в случае максимальной кучи) или меньше или равно (в случае минимальной кучи) значений своих дочерних вершин.

Кучи часто используются для создания эффективных алгоритмов сортировки, таких как heapsort, а также для создания очередей с приоритетами, которые используются во многих алгоритмах, таких как алгоритм Дейкстры для нахождения кратчайшего пути в графе.



\section{Definicja i klasy złożoności obliczeniowej – czasowej i pamięciowej.}

\subsection*{Временная сложность}

Временная сложность алгоритма определяет количество операций, которое алгоритму потребуется выполнить, в зависимости от размера входных данных. Она обычно выражается с использованием "Big O notation", которая описывает верхнюю границу временной сложности в худшем случае.

Для анализа временной сложности алгоритма нужно сначала определить, что считать "операцией". В простейшем случае это может быть любая простая операция, такая как сложение, умножение, сравнение и т.д. Затем нужно посчитать, сколько таких операций будет выполнено в худшем случае для входных данных определенного размера.

Например, простой алгоритм поиска элемента в неотсортированном массиве будет иметь временную сложность O(n), где n - это размер массива. Это потому, что в худшем случае нам придется просмотреть весь массив, чтобы найти нужный элемент.

\subsection*{Сложность по памяти}

Сложность по памяти алгоритма определяет количество памяти, которое алгоритму потребуется занять, в зависимости от размера входных данных. Как и временная сложность, сложность по памяти также обычно оценивается асимптотически и выражается с использованием нотации O большого.

Для анализа сложности по памяти нужно посчитать, сколько памяти требуется для хранения входных данных и любых дополнительных структур данных, которые используются в процессе выполнения алгоритма.

Например, простой алгоритм сортировки вставками имеет сложность по памяти O(1), потому что он сортирует массив на месте и не требует дополнительной памяти, независимо от размера входного массива.

