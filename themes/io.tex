\chapter{IO MMMM OAOAO}
\section{Standardowe metodyki procesu wytwórczego oprogramowania.}

Стандартные методологии разработки программного обеспечения предлагают структурированный подход к планированию, разработке, тестированию и управлению процессом разработки ПО. Вот некоторые из наиболее известных методологий:

\begin{itemize}
\item \textbf{Waterfall:} Это линейная и последовательная модель, где каждая стадия начинается только после завершения предыдущей. Этапы включают сбор требований, проектирование, реализацию, тестирование, развертывание и поддержку.

\item \textbf{Agile:} Agile представляет собой гибкую методологию, которая обеспечивает быструю разработку и поставку функциональности через инкрементные "спринты". Она подчеркивает коммуникацию, обратную связь и адаптивность к изменениям.

\item \textbf{Scrum:} Scrum - это форма Agile, которая включает в себя короткие фазы разработки, называемые "спринтами". Каждый спринт продолжается от одной до четырех недель и включает планирование, разработку, ревью и ретроспективу.

\item \textbf{Lean:} Lean разработка основана на принципах бережливого производства и фокусируется на устранении отходов, повышении эффективности и поставке максимальной ценности для клиента.

\item \textbf{DevOps:} DevOps является практикой, которая сосредоточена на совместной работе разработчиков и операционных команд с целью улучшить коммуникацию, сотрудничество и интеграцию, а также ускорить поставку ПО.
\end{itemize}

Важно понимать, что нет одной "лучшей" методологии для каждого проекта или организации. Подход, который следует выбирать, зависит от многих факторов, включая природу проекта, размер и навыки команды, предпочтения клиента, и т.д.

\section{Metodyki zwinne – SCRUM.}

\textbf{Scrum} является подходом к управлению проектами, который подчеркивает гибкость, быструю итерацию и коллективную ответственность. Он был первоначально разработан для проектов по разработке программного обеспечения, но впоследствии был адаптирован для различных видов командных проектов. Вот основные компоненты методологии Scrum:

\begin{itemize}
\item \textbf{Роли:}
\begin{itemize}
\item \textbf{Product Owner (Владелец продукта)}: Ответственный за определение и приоритизацию элементов бэклога продукта.
\item \textbf{Scrum Master}: Обеспечивает следование принципам и практикам Scrum. Служит в качестве моста между командой и внешними стейкхолдерами.
\item \textbf{Scrum Team (Команда)}: Группа, которая разрабатывает продукт и вместе принимает решения.
\end{itemize}
\item \textbf{Артефакты:}
\begin{itemize}
\item \textbf{Product Backlog (Бэклог продукта)}: Список требований к продукту, упорядоченных по приоритетам.
\item \textbf{Sprint Backlog (Бэклог спринта)}: Подмножество бэклога продукта, выбранное для реализации в течение следующего спринта.
\item \textbf{Increment (Инкремент продукта)}: Совокупность всех элементов бэклога продукта, реализованных в течение спринта.
\end{itemize}
\item \textbf{События:}
\begin{itemize}
\item \textbf{Sprint Planning (Планирование спринта)}: Встреча, на которой команда определяет, что будет разработано в течение следующего спринта.
\item \textbf{Daily Scrum (Ежедневный стоячий митинг)}: Короткая встреча для обсуждения прогресса и планирования работы на день.
\item \textbf{Sprint Review (Ревью спринта)}: Встреча, на которой команда показывает то, что было разработано в течение спринта.
\item \textbf{Sprint Retrospective (Ретроспектива спринта)}: Встреча после ревью спринта, где команда обсуждает, что хорошо прошло, что плохо, и как можно улучшить следующий спринт.
\end{itemize}
\end{itemize}

Основная идея Scrum - создать небольшие, самоорганизующиеся команды, которые могут быстро реагировать на изменения и динамически планировать и проводить работу.


\section{Testowanie oprogramowania.}

\textbf{Тестирование программного обеспечения} — это процесс оценки функциональности программного продукта для обнаружения различий между существующими и требуемыми условиями (то есть дефектами/ошибками/багами) и для понимания рисков использования программного продукта.

Вот некоторые ключевые аспекты тестирования программного обеспечения:

\begin{itemize}
\item \textbf{Функциональное тестирование:} Это вид тестирования, в котором ПО проверяется на соответствие его функциональным требованиям.

\item \textbf{Тестирование производительности:} Включает в себя тестирование на нагрузку, стресс-тестирование и тестирование стабильности, чтобы гарантировать, что ПО сможет работать под высокой нагрузкой и в течение продолжительного времени.

\item \textbf{Регрессионное тестирование:} Это вид тестирования, проводимого для убеждения, что внесенные изменения в код не влияют на уже существующую функциональность.

\item \textbf{Модульное тестирование:} Здесь отдельные модули программного обеспечения проверяются независимо друг от друга.

\item \textbf{Интеграционное тестирование:} Тестирует, как различные модули работают вместе, обнаруживая проблемы во взаимодействии.

\item \textbf{Системное тестирование:} Проверяет полную систему для проверки того, что весь продукт работает правильно и в соответствии со спецификациями.

\item \textbf{Приемочное тестирование:} Конечный этап тестирования, который определяет, будет ли система принята или нет. Это может включать в себя тестирование пользователя (UAT), когда реальные пользователи проверяют систему на соответствие их требованиям.
\end{itemize}

Тестирование ПО — это критически важная часть процесса разработки ПО, которая требует планирования, проектирования, построения тестов и анализа результатов.


\section{Diagramy UML. }

\textbf{Unified Modeling Language (UML)} является стандартным языком для определения и визуализации модели системы программного обеспечения. Он представляет собой набор инструментов для записи и анализа систем и используется для общения об архитектуре и дизайне программных систем. UML включает в себя несколько видов диаграмм, каждая из которых отображает различные аспекты системы:

\begin{itemize}
\item \textbf{Диаграмма классов:} Используется для визуализации статической структуры системы, показывая классы, их атрибуты и методы, а также отношения между ними.

\item \textbf{Диаграмма последовательности:} Используется для иллюстрации, как объекты взаимодействуют во времени. Она показывает, как сообщения передаются между объектами в течение определенного сценария.

\item \textbf{Диаграмма состояний:} Используется для описания поведения системы, отображая состояния объекта и переходы между ними.

\item \textbf{Диаграмма случаев использования:} Используется для представления функциональности системы с точки зрения внешних акторов и их взаимодействия со системой.

\item \textbf{Диаграмма активности:} Показывает поток управления от одной деятельности к другой в системе. Они очень похожи на диаграммы потоков данных и могут быть использованы для моделирования бизнес-процессов или рабочего процесса системы.
\end{itemize}

С помощью этих диаграмм UML может быть использован для визуализации, спецификации, конструирования и документирования артефактов системы программного обеспечения.
