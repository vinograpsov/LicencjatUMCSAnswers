\chapter{OOP}
\section{Obiekt i klasa w wybranym języku programowania zorientowanym obiektowo.}

\subsection*{Класс}

Класс в Java — это шаблон или чертёж, используемый для описания объектов. Класс определяет свойства (атрибуты) и методы, которые будут общими для всех объектов данного класса. Вот пример простого класса в Java:

\begin{lstlisting}[language=Java]
public class Car {
    // class fields
    private String color;
    private int speed;

    // Class methods
    public void setColor(String color) {
        this.color = color;
    }

    public void setSpeed(int speed) {
        this.speed = speed;
    }

    public String getColor() {
        return this.color;
    }

    public int getSpeed() {
        return this.speed;
    }
}
\end{lstlisting}

\subsection*{Объект}

Объект в Java — это экземпляр класса. Создание объекта класса называется инстанциацией. Каждый объект имеет свои собственные значения для свойств, определённых в классе. Вот как вы можете создать объект класса `Car` в Java:

\begin{lstlisting}[language=Java]
public class Main {
    public static void main(String[] args) {
        // creation obj Car
        Car myCar = new Car();

        // using methods of Car class to set and get values
        myCar.setColor("Red");
        myCar.setSpeed(70);
        System.out.println("Car color: " + myCar.getColor());
        System.out.println("Car speed: " + myCar.getSpeed());
    }
}
\end{lstlisting}


\section{Hermetyzacja, dziedziczenie i polimorfizm w programowaniu obiektowym.}

\subsection*{Герметизация}

Герметизация, или инкапсуляция, — это концепция ООП, которая скрывает детали реализации от пользователя. Если метод или переменная объявлены с модификатором private, они не могут быть доступны извне класса.

\begin{lstlisting}[language=Java]
public class Example {
    private String hiddenVar;

    public String getHiddenVar() {
        return this.hiddenVar;
    }

    public void setHiddenVar(String value) {
        this.hiddenVar = value;
    }
}
\end{lstlisting}

\subsection*{Наследование}

Наследование — это свойство, которое позволяет одному классу наследовать поля и методы другого. С помощью наследования можно создавать более общие классы, а затем расширять их для создания более специализированных классов.

\begin{lstlisting}[language=Java]
public class Vehicle {
    public void move() {
        System.out.println("Vehicle is moving");
    }
}

public class Car extends Vehicle {
    @Override
    public void move() {
        System.out.println("Car is moving");
    }
}
\end{lstlisting}

\subsection*{Полиморфизм}

Полиморфизм позволяет использовать объекты различных типов с общим интерфейсом. В Java это достигается благодаря использованию интерфейсов и наследования.

\begin{lstlisting}[language=Java]
public interface Animal {
    public void makeSound();
}

public class Cat implements Animal {
    @Override
    public void makeSound() {
        System.out.println("Meow");
    }
}

public class Dog implements Animal {
    @Override
    public void makeSound() {
        System.out.println("Woof");
    }
}
\end{lstlisting}


\section{Interfejsy i klasy abstrakcyjne w programowaniu obiektowym.}



\subsection*{Интерфейсы}

Интерфейс в объектно-ориентированном программировании — это контракт, который определяет, какие методы должны быть реализованы в классе. Интерфейсы не содержат деталей реализации методов. В Java это может выглядеть так:

\begin{lstlisting}[language=Java]
public interface Animal {
    void makeSound();
}
\end{lstlisting}

\subsection*{Абстрактные классы}

Абстрактный класс — это класс, который не может быть инстанциирован напрямую, и который обычно содержит один или несколько абстрактных методов (методов без реализации). Он служит базой для подклассов, которые должны реализовать все абстрактные методы. Пример абстрактного класса в Java:

\begin{lstlisting}[language=Java]
public abstract class Animal {
    public abstract void makeSound();

    public void eat() {
        System.out.println("The animal eats");
    }
}
\end{lstlisting}

В этом примере, `makeSound` — это абстрактный метод, который должен быть реализован в каждом конкретном подклассе класса `Animal`. В то же время, `eat` — это обычный метод с реализацией, и его можно переопределить в подклассе, если требуется.
