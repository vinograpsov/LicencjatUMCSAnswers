\chapter{XZ CZTO ETO}
\section{Modele reprezentacji wiedzy}

\textit{Модели представления знаний} используются для формализации и структурирования информации и знаний таким образом, чтобы они могли быть обработаны компьютерами. Ниже приведены некоторые общие типы моделей представления знаний.

\subsection{Семантические сети}

\textit{Семантические сети} – это графическая структура, которая используется для представления знаний в форме сети, где узлы представляют концепции, а дуги представляют отношения между концепциями.

\subsection{Базы данных}

\textit{Базы данных} обеспечивают структурированный и организованный способ хранения, управления и извлечения информации. Модели данных, такие как реляционная, иерархическая и сетевая, используются для организации данных в базах данных.

\subsection{Базы знаний}

\textit{Базы знаний} - это тип базы данных, специально разработанный для управления знаниями. Они часто используют формы логики и онтологии для представления сложных структур знаний и отношений.

\subsection{Экспертные системы}

\textit{Экспертные системы} используют базы знаний в сочетании с набором правил, чтобы моделировать решения, которые могут быть приняты экспертом в определенной области.

\subsection{Онтологии}

\textit{Онтологии} - это формальные, явные спецификации общих концепций в определенной области. Они используются для представления знаний в структурированной и стандартизированной форме и могут быть использованы для облегчения обмена информацией и интеграции данных.


\section{Mechanizmy wnioskowań}
\textit{Механизмы рассуждения} – это методы, используемые для вывода новой информации из имеющихся данных и знаний. Они играют ключевую роль в области искусственного интеллекта и науки о данных. Ниже приведены некоторые основные типы механизмов рассуждения.

\subsection{Дедуктивное рассуждение}

\textit{Дедуктивное рассуждение} – это процесс вывода конкретных выводов из общих принципов или предложений. Если все предположения верны и логика правильна, то вывод должен быть верным.

\subsection{Индуктивное рассуждение}

\textit{Индуктивное рассуждение} – это процесс формирования общих утверждений на основе конкретных наблюдений или примеров. Оно обычно используется для создания новых теорий или гипотез.

\subsection{Абдуктивное рассуждение}

\textit{Абдуктивное рассуждение} – это процесс формирования наиболее вероятного объяснения для определенного набора наблюдений. Оно часто используется в области диагностики и обнаружения причинно-следственных связей.

\subsection{Аналоговое рассуждение}

\textit{Аналоговое рассуждение} - это процесс применения знаний или опыта, полученного в одном контексте, к новому контексту на основе обнаруженных сходств.

\subsection{Статистическое рассуждение}

\textit{Статистическое рассуждение} - это использование теории вероятностей и статистического анализа для формирования выводов на основе данных.

\subsection{Машинное обучение}

\textit{Машинное обучение} - это подраздел искусственного интеллекта, который использует механизмы рассуждения для создания моделей, способных обучаться и делать прогнозы на основе данных. Это включает в себя различные типы обучения, такие как обучение с учителем, обучение без учителя и обучение с подкреплением.

```latex
\subsection{Искусственные нейронные сети}

\textit{Искусственные нейронные сети} - это модели, которые имитируют структуру и функцию биологических нейронных сетей. Они способны обучаться и адаптироваться к новым данным, что делает их мощным инструментом для многих задач машинного обучения.


\section{ Sposoby cyfrowej reprezentacji liczby całkowitej i rzeczywistej. }

\subsection{Целые числа}

Целые числа обычно представляются в виде двоичных чисел. Вот некоторые из основных систем представления:

\begin{itemize}
\item \textbf{Двоичная система}: Числа представлены в виде последовательности двоичных цифр (0 и 1).
\item \textbf{Дополнительный код}: Для представления отрицательных чисел обычно используется дополнительный код. В дополнительном коде отрицательное число представляется как "дополнение" соответствующего положительного числа.
\item \textbf{Двоично-десятичная кодировка}: В этом случае каждая десятичная цифра представляется своим двоичным эквивалентом. Это обычно используется в финансовых вычислениях, где требуется высокая точность.
\end{itemize}

\subsection{Вещественные числа}

Вещественные числа обычно представляются в виде чисел с плавающей запятой. Наиболее распространенный стандарт представления чисел с плавающей запятой - это IEEE 754. 

\begin{itemize}
\item \textbf{Стандарт IEEE 754}: Этот стандарт определяет, как представлять вещественные числа в двоичной форме и как выполнять арифметические операции с числами с плавающей запятой. Он включает в себя специальные значения, такие как "не число" (NaN), "плюс бесконечность" и "минус бесконечность".
\end{itemize}

